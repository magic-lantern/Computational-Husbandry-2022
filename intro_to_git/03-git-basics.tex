\section{Basic Git Use}
\begin{frame}[t]
  \frametitle{Outline}
  \framesubtitle{Why? and little bit of How?}
  \tableofcontents[currentsection,hideallsubsections] 
\end{frame}

\begin{frame}[t]{What will be covered in this section?}
  \begin{itemize}
    \item Working on a local repository.
    \item Working with a remote.
    \item Working with other people with a common remote.
    \item[]
    \item Read Chapter 2 of {\it Pro Git} for more information.
    \item[]
    \item Examples will be done using the command line.
  \end{itemize}
\end{frame}


\begin{frame}[t]
  \frametitle{Working on a Local Repository}
  \framesubtitle{Objectives:}
    We will learn the following Git verbs:
  \begin{itemize}
    \item {\tt init}
    \item {\tt status}
    \item {\tt add}
    \item {\tt commit}
    \item {\tt diff}
    \item {\tt log}
    \item {\tt checkout} 
    %\item {\tt reset}
    \item {\tt mv} 
    \item {\tt rm}
  \end{itemize}
  
  Other helpful things:
  \begin{itemize}
    \item {\tt .gitignore} files
  \end{itemize}
\end{frame}

\begin{frame}[t]
  \frametitle{Working with a Remote}
  \framesubtitle{Objectives:}
  Remote Hosting Options
  \begin{itemize}
    \item \url{github.com}, \url{gitlab.com}, \url{bitbucket.com}, and many others.
    \item Read the end user agreements. 
    \item Public or Private repo: data is stored on a public server.
    \item You can run remote servers behind firewall
  \end{itemize}

  Authentication:
  \begin{itemize}
    \item https
      \begin{itemize}
        \item Okay, quick and simple
        \item May require you provide username/password for each push, fetch
        \item Not supported if you have 2FA set up
      \end{itemize}
    \item ssh
      \begin{itemize}
        \item Easy to set up
        \item Better security, works within 2FA
        \item I like to have a unique public/private key pair for each computer I work on.  If a
          machine is compromised I can protect all my work by revoking that
          specific key.
      \end{itemize}
  \end{itemize}

  We will learn the following Git verbs:
  \begin{itemize}
    \item {\tt remote}
    \item {\tt push}
    \item {\tt clone}
    \item {\tt fetch}
    \item {\tt merge}
    \item {\tt pull} (this is short-hand for {\tt fetch} and {\tt merge} in one
      step)
  \end{itemize} 
\end{frame}

\begin{frame}[t]{Objectives:}
  \frametitle{Working with Others and Remotes}
  Owners of the repo can set read/write permissions.

  We will learn the following Git verbs:
  \begin{itemize}
    \item {\tt merge}
    \item {\tt diff-tools}
  \end{itemize} 
\end{frame}

